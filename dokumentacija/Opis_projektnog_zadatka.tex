\chapter{Opis projektnog zadatka}
		
			
			\textbf{\textit{dio 1. revizije}}\\
			
			{\color{red}\textit{Na osnovi projektnog zadatka detaljno opisati korisničke zahtjeve. Što jasnije opisati cilj projektnog zadatka, razraditi problematiku zadatka, dodati nove aspekte problema i potencijalnih rješenja. Očekuje se minimalno 3, a poželjno 4-5 stranica opisa.	Teme koje treba dodatno razraditi u ovom poglavlju su:}
			\begin{packed_item}
				\item \textit{potencijalna korist ovog projekta}
				\item \textit{postojeća slična rješenja (istražiti i ukratko opisati razlike u odnosu na zadani zadatak). Dodajte slike koja predočavaju slična rješenja.}
				\item \textit{skup korisnika koji bi mogao biti zainteresiran za ostvareno rješenje.}
				\item \textit{mogućnost prilagodbe rješenja }
				\item \textit{opseg projektnog zadatka}
				\item \textit{moguće nadogradnje projektnog zadatka}
			\end{packed_item}
			
			\textit{Za pomoć pogledati reference navedene u poglavlju „Popis literature“, a po potrebi konzultirati sadržaj na internetu koji nudi dobre smjernice u tom pogledu.}
			\eject
		}
		
		Cilj ovog projekta napraviti je web aplikaciju za dojavu oštećenja i drugih problema na cestama, parkovima, javnim ustanovama i drugim javnim mjestima u svrhu olakšavanja dojave, kategorizacije te u konačnici rješenja prijavljenih problema.
		
		Jedinstveni sustav za praćenje problema uvelike olakšava građanima da prijavljuju probleme na koje naiđu. S obzirom na velik broj javih ustanova i njihovih odjeljaka koji se bave različitim problemima, teško je pratiti tko je nadležan za koju vrstu problema i nad kojim područjem. Sustav za dojavu problema bi automatski odredio nadležno tijelo prema kategoriji prijave i drugim parametrima te proslijedio prijavu na obradu. Ključno je da proces prijave bude jednostavan i brz kako ne bi upao u istu zamku komplicirane birokracije trenutnog sustava.
		
		Naš sustav će omogućiti građanima da lako prijave razne vrste komunalnih i drugih problema. Prikaz liste i prikaz na karti daju jasan pregled prijavljenih problema i uz njih priloženih slika i opisa te trenutni status obrade. U slučaju da je za isti problem podneseno više prijava, sustav pita korisnika želi li grupirati svoju prijavu s drugom koja je lokacijski i vremenski bliska. U svrhu dodatnog olakšavanja podnošenja prijave, od građana se ne zahtjeva da su prijavljeni na (niti da posjeduju) korisnički račun za podnošenje prijave. U tom slučaju status prijave mogu pratiti jedinstvenim identifikatorom prijave ili pronalaženjem prijave na web aplikaciji.
		
		Nakon što sustav proslijedi prijavu u nadležni gradski ured, službenik tog ureda može izabrati problem i krenuti raditi na rješenju. U aplikaciji može promijeniti status problema i vidjeti informacije iz poslanih prijava.
		
		Jedan primjer takvog sustava koji se aktivno upotrebljava u Hrvatskoj je "Gradsko Oko". Projekt je pokrenut u kolovozu 2017. godine u svrhu prijave komunalnih problema na području grada Bjelovara, a od tad se proširio na još 11 gradova i općina te na prijavu problema na moru. Taj je projekt vrlo sličan našem, ali razlikuje se u nekoliko točaka: ne dozvoljava anonimne prijave, ne sadrži mogućnost pregleda statistike, ne provjerava sličnost vremenski bliskih prijava u svrhu grupiranja.
		
		
		
	