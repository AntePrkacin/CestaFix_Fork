\chapter{Zaključak i budući rad}
		
		Zadatak našeg tima bio je razvoj aplikacije za prijavu štete javnih površina. Aplikacija je trebala olakšati 
		prijavu oštećenja javnih površina i cesta u gradovima i Provedba projekta bila je u dvije faze.

		Prva faza projekta uključivala je formiranje tima za razvoj aplikacije, dodjelu projektnog zadatka i intenzivan
		rad na dokumentiranju zahtjeva. Kvalitetnom provedbom prve faze uvelike je olakšan daljnji rad na razvoju 
		osmišljene aplikacije. Izrađeni obrasci i dijagrami (obrasci uporabe, sekvencijski dijagrami, model baze podataka,
		dijagrami razreda) omogućili su razvojnim timovima za \textit{frontend} i \textit{backend} lakši rad i organizaciju.
		Izrada vizualnog koncepta aplikacije uštedila je mnogo vremena u drugoj fazi razvoja, gdje je služila kao idejni 
		temelj za razvoj. Tijekom prve faze projekta ostvarene su osnovne funkcionalnosti aplikacije.

		Druga faza projekta bila je kraća, ali puno intenzivnija u razvoju aplikacije. Nedostatak iskustva članova u izradi
		sličnih projekata bio je izazov, no kroz projekt članovi tima imali su priliku upoznati i savladati razne alate i 
		programske jezike kako bi uspješno realizirali projekt. Uz tehničke vještine članovi su imali priliku razvijati 
		organizacijske vještine. Tijekom druge faze izrađeni su novi dijagrami (dijagram stanja, dijagram aktivnosti, dijagram
		komponenti, dijagram razmještaja) te unaprijeđeni već postojeći dijagrami (model baze podataka, dijagrami razreda).
		Dobro izrađen kostur projekta uštedio nam je mnogo vremena prilikom izrade aplikacije. Tijekom druge faze projekta ostvarena 
		je većina funkcionalnosti aplikacije.

		Tijekom projekta uspjeli smo realizirati većinu zamišljenih funkcionalnosti. One se odnose na funkcionalnosti za 
		neregistriranog korisnika i registriranog korisnika, koji se dijeli na građanina i službenika gradskog ureda. Time 
		smo ostvarili sve tražene funkcionalnosti zadane u projektnom zadatku. Iznimka od toga je obrazac uporabe UC20 koji se 
		odnosi na ponovno postavljanje zaboravljene lozinke. Spomenuti obrazac uporabe nije zamišljen u projektnom zadatku nego
		kao ideja za korisnu funkcionalnost u aplikaciji, stoga zbog vremenske ograničenosti preostaje za budući rad na aplikaciji.
		Funkcionalnosti administratora također nam preostaju za budući rad. Konkretno je riječ o obrascima uporabe UC04, UC05,
		UC06, UC18, UC19, UC21, UC22 i UC23. U trenutnoj verziji aplikacije one se ostvaruju ručno u bazi podataka. 

		Informiranost članova o napretku projekta ostvarili smo korištenjem aplikacije Discord i sastancima uživo. Svi 
		članovi pokazali su zainteresiranost za projekt čime je rad bio ugodan i bez organizacijskih problema. Zadovoljni smo s
		realiziranom aplikacijom i zajedničkim radom na projektu.

		\eject 