\chapter{Dnevnik promjena dokumentacije}

\textbf{\textit{Kontinuirano osvježavanje}}\\


\begin{longtblr}[
	label=none
	]{
	width = \textwidth,
	colspec={|X[2]|X[13]|X[5]|X[4]|},
	rowhead = 1
	}
	\hline
	\textbf{Rev.} & \textbf{Opis promjene/dodatka}                                                                                            & \textbf{Autori} & \textbf{Datum} \\[3pt] \hline
	0.1           & Preuzet predložak.                                                                                                        & Sara Podvorec   & 28.10.2023.    \\[3pt] \hline
	0.2           & Dodane osnovne informacije. Napisan početak opisa projekta. Crvenom bojom označeni dijelovi teksta koje treba izmijeniti. & Jan Murić       & 28.10.2023.    \\[3pt] \hline
	0.3           & Nadopuna opisa projekta.                                                                                                  & Sara Podvorec   & 29.10.2023.    \\[3pt] \hline
	0.4           & Nadopuna osnovnih informacija. Nadupuna dnevnika sastajanja.                                                              & Mateo Jakšić    & 30.10.2023.    \\[3pt] \hline
	0.5           & Stiliziranje opisa projekta.                                                                                              & Sara Podvorec   & 30.10.2023.    \\[3pt] \hline
	0.6           & Nadopuna dnevnika sastajanja.                                                                                             & Mateo Jakšić    & 31.10.2023.    \\[3pt] \hline
	0.7           & Nadopuna opisa projektnog zadatka i funkcionalnih zahtjeva.                                                               & Sara Podvorec   & 31.10.2023.    \\[3pt] \hline
	0.8           & Nadopuna literature i funkcionalnih zahtjeva.                                                                             & Mateo Jakšić    & 1.11.2023.     \\[3pt] \hline
	0.9           & Napisan dio obrazaca uporabe                                                                                              & Mateo Jakšić    & 1.11.2023.     \\[3pt] \hline
	0.10          & Nadopuna obrazaca uporabe                                                                                                 & Mateo Jakšić    & 2.11.2023.     \\[3pt] \hline
	0.11          & Nadopuna obrazaca uporabe. Dodani dijagrami obrazaca uporabe.                                                             & Mateo Jakšić    & 6.11.2023.     \\[3pt] \hline
	0.12          & Dodani sekvencijski dijagrami. Dodani ostali zahtjevi. Nadopuna dnevnika sastajanja.                                      & Mateo Jakšić    & 10.11.2023.    \\[3pt] \hline
	0.13          & Dodane osnovne informacije o arhitekturi. Nadopuna opisa baze podataka.                                                   & Sara Podvorec   & 12.11.2023.    \\[3pt] \hline
	0.14          & Dodan opis pojedinih entiteta baze podataka.                                                                              & Sara Podvorec   & 13.11.2023.    \\[3pt] \hline
	0.15          & Nadopuna dnevnika sastajanja. Nadopuna opisa i kreiranje ERD dijagrama baze podataka.                                     & Mateo Jakšić    & 14.11.2023.    \\[3pt] \hline
	0.16          & Dodatan ER dijagram baze podataka                                                                                         & Mateo Jakšić    & 15.11.2023.    \\[3pt] \hline
	0.17          & Dodan opis arhitekture. Dodan vizualni prikaz arhitekture.                                                                & Sara Podvorec   & 16.11.2023.    \\[3pt] \hline
	0.18          & Dodani dijagrami razreda i njihovi opisi. Nadopuna literature.                                                            & Mateo Jakšić    & 16.11.2023.    \\[3pt] \hline
\end{longtblr}


{\textit{Moraju postojati glavne revizije dokumenata 1.0 i 2.0 na kraju prvog i drugog ciklusa. Između tih revizija mogu postojati manje revizije već prema tome kako se dokument bude nadopunjavao. Očekuje se da nakon svake značajnije promjene (dodatka, izmjene, uklanjanja dijelova teksta i popratnih grafičkih sadržaja) dokumenta se to zabilježi kao revizija. Npr., revizije unutar prvog ciklusa će imati oznake 0.1, 0.2, …, 0.9, 0.10, 0.11.. sve do konačne revizije prvog ciklusa 1.0. U drugom ciklusu se nastavlja s revizijama 1.1, 1.2, itd.}}